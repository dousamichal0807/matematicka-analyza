\section{Definice}
\label{sec:derivace-definice}

\emph{Derivace}, někdy též \emph{diferenciál}, vyjadřuje, v~jaké míře jistá
funkce~$f$ roste či klesá. Derivace je takový prostředek, který umožňuje, pokud
možno, pro všechna ${x \in \fndom(f)}$ určit, jakým způsobem funkce roste či
klesá v~dané hodnotě~$x$. Nejdříve začneme myšlenku derivace formovat intuitivně
a~následně ji zformalizujeme v~sekci~\ref{subsec:derivace-definice-formal}.

\subsection{Derivace pomocí tečen}
\label{subsec:derivace-pomoci-tecen}

Určit, jak funkce roste či klesá, je dost obtížné.
Máme ale přímku.
Je-li přímka grafem jisté funkce ${g(x) = kx + q}$, za~míru růstu resp.~poklesu
funkční hodnoty můžeme označit koeficient~$k$, tj. směrnici přímky, která je grafem
funkce~$g$.
Grafem funkce je ale většinou křivka, například parabola, sinusoida, hyperbola či
kuboida\footnote{\emph{Kuboida} je křivka, která je grafem kubické funkce.
\emph{Kubická funkce} je funkce, jejíž funkční předpis je polynom třetího stupně.}.

To ale ničemu nevadí -- ke~grafu funkce určíme tečnu -- a~tečna je již přímka.
Můžeme tedy prohlásit, že funkce~$f$ v~určité hodnotě $x$ roste či klesá stejnou
mírou, jako tečna v~tom stejném bodě.

Přesněji: pro každou hodnotu ${x \in \fndom(f)}$ se bude hodnota derivace v~daném
bodě~$x$ rovnat směrnici tečny $t$ ke~grafu funkce~$f$ s~bodem dotyku $T[x, f(x)]$.
To samozřejmě jen tehdy, pokud tečna~$t$ existuje a~jsme schopni určit její směrnici.

\begin{exercise}[Tečna bez směrnice]
    Kdy můžeme mít tečnu, jejíž směrnici nebudeme schopni určit? Jak musí být taková
    tečna, či obecně přímka, orientovaná? Existuje funkce a~k~ní určitý bod na~grafu
    funkce, kde jsme schopni zkonstruovat tečnu, u~které směrnici nejsme schopni
    určit? Pokud ano, uveďte alespoň jeden příklad takové funkce a~příslušného bodu
    na~grafu.
\end{exercise}

Také si všimněte, že hodnotám ${x \in \fndom(f)}$ určíme maximálně jedno reálné
číslo, které určuje strmost grafu funkce~$f$.
Proto prohlásíme, že výstupem derivace bude nějaká jiná funkce, kterou označíme
jako~$f'$, jejíž funkční hodnota reprezentuje právě strmost grafu původní funkce~$f$.

\begin{figure}[ht!]
    \centering
    \begin{tikzpicture}
    %Barvy
    \colorlet{tangentLine}{accentColor}
    \colorlet{point}{accentColor!50!black}

    % Svislá mřížka
    \foreach \x in {-2,...,2} {
        \ifnum \x = 0 {} \else {
            \draw (\x, 0) node[anchor=north, inner sep=1mm, color=lightgray] {$\x$};
            \draw[dash pattern=on 0 off 0.5mm, line cap=round] (\x, -2.5) -- (\x, 2.5);
            \draw (\x, 0.05) -- (\x, -0.05);
        } \fi
    }
    % Vodorovná mřížka
    \foreach \y in {-2,...,2} {
        \ifnum \y = 0 {} \else {
            \draw (0, \y) node[anchor=east, inner sep=1mm, color=lightgray] {$\y$};
            \draw[dash pattern=on 0 off 0.5mm, line cap=round] (-2.5, \y) -- (2.5, \y);
            \draw (0.05, \y) -- (-0.05, \y);
        } \fi
    }
    % Osa x a y
    \draw[->] (-2.5, 0) -- (2.5, 0);
    \draw (2.5, 0) node[anchor=south east, inner sep=1mm] {$x$};
    \draw[->] (0, -2.5) -- (0, 2.5);
    \draw (0, 2.5) node[anchor=north west, inner sep=1mm] {$y$};

    % Graf
    \draw[color=point] (-2.3, -2) node[anchor=west, inner sep=1mm] {$f(x)$};
    \directlua{figDerivTecny = require("chapters/derivace/fig-tecny")}
    \directlua{figDerivTecny.graph()}

    % Tečny a body dotyku

    \draw[thick, color=tangentLine] (-1.63, 1.165) -- (-2.25, -1.625);
    \draw[color=tangentLine] (-2.25, -1.625) node[anchor=south east, inner sep=0.5mm] {$t_1$};
    \fill[color=point] (-2, -0.5) circle (0.05);
    \draw[color=point] (-2, -0.5) node[anchor=north west, inner sep=0.5mm] {$T_1$};

    \draw[thick, color=tangentLine] (-2, 0.125) -- (-1, 2);
    \draw[color=tangentLine] (-1, 2) node[anchor=south east, inner sep=0.5mm] {$t_2$};
    \fill[color=point] (-1.5, 1.0625) circle (0.05);
    \draw[color=point] (-1.5, 1.0625) node[anchor=north west, inner sep=0.5mm] {$T_2$};

    \draw[thick, color=tangentLine] (-2.25, 1.5) -- (0.25, 1.5);
    \draw[color=tangentLine] (0.25, 1.5) node[anchor=south, inner sep=1mm] {$t_3$};
    \fill[color=point] (-1, 1.5) circle (0.05);
    \draw[color=point] (-1, 1.5) node[anchor=south west, inner sep=0.5mm] {$T_3$};

    \fill[color=point] (0, 0.5) circle (0.05);
    \draw[color=point] (0, 0.5) node[anchor=west, inner sep=1mm] {$T_4$};

    \draw[thick, color=tangentLine] (-0.25, -0.5) -- (2.25, -0.5);
    \draw[color=tangentLine] (-0.25, -0.5) node[anchor=south, inner sep=1mm] {$t_5$};
    \fill[color=point] (1, -0.5) circle (0.05);
    \draw[color=point] (1, -0.5) node[anchor=north east, inner sep=0.5mm] {$T_5$};

    \draw[thick, color=tangentLine] (2, 0.875) -- (1, -1);
    \draw[color=tangentLine] (1, -1) node[anchor=west, inner sep=.15cm] {$t_6$};
    \fill[color=point] (1.5, -0.0625) circle (0.05);
    \draw[color=point] (1.5, -0.0625) node[anchor=south east, inner sep=0.5mm] {$T_6$};

    \draw[thick, color=tangentLine] (1.68, 0.06) -- (2.22, 2.49);
    \draw[color=tangentLine] (2, 1.5) node[anchor=north west, inner sep=0.5mm] {$t_7$};
    \fill[color=point] (2, 1.5) circle (0.05);
    \draw[color=point] (2, 1.5) node[anchor=east, inner sep=1mm] {$T_7$};
\end{tikzpicture}

    \caption{Ukázka, jak by derivace měla fungovat}
    \label{fig:derivace-tecny}
\end{figure}

Jak by náš koncept měl fungovat, je demonstrováno na~obrázku~\ref{fig:derivace-tecny},
kde zvolíme body $T_1[x_1, y_1]$ až $T_7[x_7, y_7]$ ležící na~grafu funkce
${f(x) = \frac{1}{2} x^3 - \frac{3}{2} x + \frac{1}{2}}$.
Zvolené body jsou body dotyku tečen $t_1$ až $t_7$ se~směrnicemi $k_1$ až $k_7$.
Proto na~grafu funkce~$f'$, která je derivací funkce~$f$, budou ležet body
$A_1[x_1, k_1]$ až $A_7[x_7, k_7]$.

Protože jsme již získali jistou intuici, budeme pokračovat v~budování teorie tím, že
vyslovíme několik lemmat, které se derivace týkají.

\begin{lemma}[Spojitost grafu]
    Derivaci~$f'(x)$ můžeme být schopni určit pouze na~spojitých úsecích grafu, vyjma
    krajních hodnot spojitého úseku a~hodnot, kde se graf zalomí.

    \begin{proof}
        Na~krajních bodech spojitého úseku nejsme schopni určit tečnu, protože nevíme,
        jestli krajní bod našeho úseku není bodem, kde se náš graf zalomí.
        V~místě, kde se graf zalomí, nejsme schopni určit derivaci funkce, protože
        hodnota derivace se z~levé strany blíží k~hodnotě, která je rozdílná
        od~hodnoty, ke~které se blíží derivace z~pravé strany.
        Jako příklad můžeme uvést funkci~${f(x) = |x|}$ v~bodě~${x = 0}$, kde je
        derivace zleva rovna~$-1$ a~zprava~$+1$.
        Proto nejsme v~takovém bodě schopni určit tečnu.
    \end{proof}
\end{lemma}

\begin{lemma}
    Směrnice tečny určuje, jestli graf funkce roste či klesá:
    \begin{itemize}
        \item Je-li směrnice kladná, graf zde určitě roste.
        \item Analogicky je-li směrnice záporná, graf zde určitě klesá.
        \item Je-li tečna kolmá k~ose $x$, tj. rovnoběžná s~osou $y$, nejsme
        schopni tečnu zapsat pomocí lineární funkce, jejíž grafem je tečna
        sama, nejsme schopni určit míru růstu a derivace zde bude nedefinovaná.
        \item Je-li tečna v~určitém bodě grafu vodorovná s~osou $x$, směrnice
        musí být nulová. Pozor -- to však ještě neznamená, že je v~blízkosti
        tohoto bodu funkce kostantní (nalezení takových funkcí nechávám jako
        úlohu).
    \end{itemize}
\end{lemma}

\begin{lemma}
    Nechť $\mathcal{F}$ je množina všech funkcí ${f: \reals \mapsto \reals}$. Poté se
    na~derivaci můžeme dívat jako na~zobrazení množiny $\mathcal{F}$ do~množiny
    $\mathcal{F}$.

    \begin{proof}
        To je dáno tím, že derivace je operace, jejíž vstupem může být jakákoliv
        funkce ${f \in \mathcal{F}}$ a~výstupem je jiná funkce ${f' \in \mathcal{F}}$.
        I~kdyby byla funkce~$f$ tvořena pouze izolovanými body, výstupem je stále
        funkce~$f'$, jejíž ${\fndom(f') = \fnran(f') = \emptyset}$.
    \end{proof}
\end{lemma}

\begin{lemma}
    Je-li $f'$ derivací funkce~$f$ a~$\fndom(f)$, platí, že
    ${\fndom(f') \subset \fndom(f)}$.

    \begin{proof}
        To vyplývá z~faktu, že funkci $f'$ přiřazujeme hodnotu pro určité
        ${x \in \fndom(f)}$, pokud vůbec, pouze pokud je $f(x)$ definované.
    \end{proof}
\end{lemma}

\begin{lemma}
    Nechť existují funkce~$f$ a~$g$, kde ${g(x) = f(x) + q}$
    a~${q \in \reals}$ je konstanta. Poté platí, že
    \begin{equation*}
        \fracdiff{f}{x} = \fracdiff{g}{x}
    \end{equation*}

    \begin{proof}
        Přičtením konstanty $q$ k~funkci~$f$ se graf funkce~$f$ posune podle
        osy~$y$. To však nezmění směrnici jakékoliv tečny ke~grafu funkce, kterou
        jsme schopni zkonstruovat. Každá tečna se posune stejným způsobem, jako
        původní funkce~$f$. Protože tečna se posouvá pouze svisle, nezmění se
        $x$-ová souřadnice tečného bodu. Protože se směrnice tečny se při posunu
        nezmění, ani se nezmění $x$-ová souřadnice tečného bodu, nemůže se změnit
        ani derivace funkce.
    \end{proof}
\end{lemma}

\begin{exercise}
    Nalezněte alespoň jednu funkci, jejíž derivace v~určité hodnotě $x$ je nulová,
    tj. směrnice tečny grafu v~daném bodě je nulová, a~přesto nemůžeme v~sebemenším
    okolí prohlásit funkci jako konstantní. Nápověda: Začněte od~těch nejjednodušších
    funkčních předpisů, které znáte -- určitě i~tam najdete takové případy.
\end{exercise}

\subsection{Formální zápis}
\label{subsec:derivace-formalni-zapis}

Nakonec bychom si řekli, jak vztah, že funkce~$f'$ je derivací (diferenciálem) funkce
$f$, zapsat. Vztah mezi funkcemi~$f$ a~$f'$ budeme zapisovat tímto způsobem:
\begin{equation*}
    \fracdiff{f}{x} = f'
\end{equation*}
Proč tento velmi zvláštní zápis? To rozebereme v~následujícím oddílu. Zastavme se
však na chvíli.

Prozatím jsme se omezovali jen na~funkci, kterou jsme označili většinou jako $f$,
a~řekli, že její derivace bude označená stejně, ale bude tam mít navíc čárku,
např.~$f'$. Pokud však budeme derivovat spíše výraz, než-li funkci, bude lepší, když
použijeme nějaký vhodný přímý zápis, než abychom pokaždé museli říct, že výraz je
funkčním předpisem nějaké funkce. Proto by se nám hodil způsob, jak zapsat derivaci
výrazu (který funkce stejně představuje). Existuje hned několik zavedených způsobů.
\begin{align*}
    \fracdiff{}{x} \left( \frac12 x - 3 \right) &= \frac12 &
    \fracdiff{\left( \frac12 x - 3 \right)}{x}  &= \frac12 &
    \left( \frac12 x - 3 \right)'               &= \frac12
\end{align*}
Všechny tři formy zápisu vyjadřují, že diferenciál výrazu ${\frac12 x - 3}$
v~závislosti na~$x$ je konstanta rovna $\frac12$. Nejčastěji se využívá možnost vlevo
-- umožňuje přehlednější zápis i mnohem složitějších výrazů. Možnost vpravo se
nevyužívá, je-li ve vzorci více proměnných (resp.\ argumentů funkce), na~jejichž
závislosti jsme schopni derivovat.

\subsection{Derivace jako podíl dvou malých změn}
\label{subsec:derivace-jako-podil}

Mějme příklad, ve~kterém budeme chtít vědět okamžité zrychlení auta v~daném časovém
momentě. Začněme však průměrným zrychlením. Fyzika v~případě změny (skoku) hodnoty
jisté proměnné používá velké řecké písmeno delta, např. $\Delta v$ by mohlo
představovat změnu rychlosti a~$\Delta t$ změnu času, tj. délku jistého časového
úseku. Proto pro~průměrné zrychlení auta určitého měřeného časového úseku, označme
jej $a_{\text{avg}}$,  bychom napsali:

\begin{equation*}
    a_{\text{avg}} = \frac{\Delta v}{\Delta t}
\end{equation*}

Konkrétně, pokud by auto zrychlilo o~7,2~km/h~=~2~m/s za půl vteřiny, platilo by, že
${\Delta v~= 2 \text{ m}/\text{s}}$ a~${\Delta t = 0,5 \text{ s}}$. Proto by průměrné
zrychlení během takové doby bylo
\begin{equation*}
    a_{\text{avg}}
    = \frac{\Delta v}{\Delta t}
    = \frac{2 \text{ m} \cdot \text{s}^{-1}}{0,5 \text{ s}}
    = 4 \text{ m} \cdot \text{s}^{-2}\text{.}
\end{equation*}

Zároveň, pokud auto zpomalí, rozdíl rychlostí je záporný, takže i~zrychlení je
záporné. Nicméně se však jedná o~průměrné zrychlení. Pokud cheme okamžité zrychlení,
musíme učinit pár úprav. Musíme změřit rychlost v~měřeném časovém úseku pokud možno
co nejvícekrát, aby průměrné zrychlení v~době mezi dvěma sousedními měřeními
rychlosti se co nejvíce blížilo okamžitému zrychlení.

To nám umožňuje zavést okamžité zrychlení jako funkci závislou na~čase, zrovna tak
i~okamžitou rychlost. Snažili bychom se funkční předpisy pro okamžitou rychlost
$v(t)$ a~okamžité zrychlení~$a(t)$ vyjádřit co nejpřesnějším funkčním předpisem
na~základě naměřených dat. Funkce~$v(t)$ a~$a(t)$ budou tedy spojité a~pomocí tečen
jsme schopni určit graf derivace obou funkcí.

Ale zároveň si povšimněme, že pokud budou naše měření přesnější a~přesnější, poté se
každá přímka vedoucí skrz jakékoliv dva sousední body bude stále více a~více
přibližovat tečně ke~grafu funkce -- a~toho využijeme.

Označme si nepatrnou změnu
rychlosti jako $\diff{v}$ místo $\Delta v$. Ta bude kladná při zrychlování
a~záporná při zpomalování. Velmi malý časový úsek mezi měřeními jako $\diff{t}$
místo $\Delta t$. Tehdy bude okamžité zrychlení v~daném čase $t$ rovno
\begin{equation*}
    a(t) = \fracdiff{v}{t}(t)
\end{equation*}
a~tento poměr velmi malých hodnot bude velmi dobře odhadovat směrnici tečny a~tudíž
i~hodnotu derivace funkce $v(t)$. Z~toho vyplývá, že derivací funkce $v(t)$ je tedy
funkce $a(t)$. Nyní již stačí tuto myšlenku formalizovat. Předtím ale ještě pár úloh.

\begin{exercise}
    Mějme funkci $s(t)$ určující, jakou vzdálenost jsme ujeli za~určitý čas. Jaká
    funkce (veličina) je derivací funkce~$s(t)$?
\end{exercise}

\begin{exercise}
    Pohybová (kinetická) energie je závislá na~rychlosti pohybu daného tělesa, rovněž
    jako hybnost. Hmotnost tělesa se standardně nemění. Jaký vztah je mezi kinetickou
    energií a~hybností? V~případě, že si již vzorce nepamatujete či je neznáte,
    můžete využít internetu. Nápověda: Nakreslete si graf hybnosti a~kinetické
    energie v~závislosti na~rychlosti pro předem danou hmotnost do~jedné soustavy
    souřadnic.
\end{exercise}

\subsection{Formální definice}
\label{subsec:derivace-definice-formal}

Opět máme funkci~$f$, jejíž derivací je funkce~$f'$. Abychom formálně definovali
derivaci musíme nejlépe sestrojit výraz, který je roven~$f'$. K~tomu potřebujeme dvě
hodnoty na~ose $x$, z~nichž spočítáme funkční hodnoty a~$f'$ bude rovno podílu
rozdílu funkčních hodnot a rozdílu argumentů (těch dvou hodnot na~ose $x$). Přitom
ale potřebujeme vyjádřit, čím blíže jsou ty dvě hodnoty na~ose $x$, tím více se
k~derivaci blížíme. K~tomu využijeme limitu:

\begin{definition}
    Mějme funkci~$f$ definovanou v~určitém okolí ${\mathcal{U}(x; \epsilon)}$, kde
    ${\epsilon \in \positive\reals}$. Existuje-li limita
    \begin{equation*}
        L = \lim_{a \to x} \frac{f(x) - f(a)}{x - a}
    \end{equation*}
    a~její hodnota je konečná, limita~$L$ je funkční hodnota derivace funkce~$f$
    v~dané hodnotě~$x$.
\end{definition}

Tato definice je sice správná, ale není v~ní příliš vidět ten náš malý rozdíl mezi
dvěma hodnotami na~ose $x$ -- v~našem případě proměnných $x$ a~$a$. Proto určíme
rozdíl $h = a - x$ a~přepíšeme zlomek tak, abychom nepoužili proměnnou~$a$.
Po~určitých úpravách dostaneme odlišný výraz:

\begin{definition}
    Mějme funkci~$f$ definovanou v~určitém okolí ${\mathcal{U}(x; \epsilon)}$, kde
    ${\epsilon \in \reals^{+}}$. Existuje-li limita
    \begin{equation*}
        L = \lim_{h \to 0} \frac{f(x + h) - f(x)}{h}
    \end{equation*}
    a~její hodnota je konečná, limita~$L$ je funkční hodnota derivace funkce~$f$
    v~dané hodnotě~$x$.
\end{definition}

S~tímto způsobem definice se zřejmě setkáte častěji. Mějte ale na~paměti, že žádná
z~definic není \emph{správnější} či \emph{přesnější}.
