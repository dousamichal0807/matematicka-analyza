\section{Diferenciál mocninné funkce}
\label{sec:diferencial-mocninne-funkce}

Pokračujme mocninnými funkcemi (tj. funkcemi ve~tvaru $f(x) = ax^m$). Takovou funkci
zderivujeme tak, že ji nejprve vynásobíme exponentem mocniny a~následně exponent
snížíme o~jedničku. Tento postup platí pro exponent v~oboru všech reálných čísel.
Tomuto pravidlu říkáme \emph{řetězové pravidlo}. Formálněji jsme schopni výrokem
zapsat toto pravidlo:
\begin{equation*}
    \forall a, m \in \reals :
    \fracdiff{}{x} \left(ax^m\right) = amx^{m - 1}
\end{equation*}

Zkusme nějaký příklad. Mějme funkci $f(x) = 3x^5$. Tu chceme zderivovat. Nejprve tedy
celý výraz vynásobíme exponentem. Dostaneme tedy výraz $15x^5$. Nyní stačí
od~exponentu odečíst jedničku. Poté tedy vznikne funkce $f'(x) = 15x^4$, která je
derivací funkce $f(x)$, tedy můžeme zapsat:
\begin{equation*}
    \fracdiff{}{x} \left( 3x^5 \right) = 15x^4.
\end{equation*}

\begin{exercise}
    Spočítejte diferenciál (derivaci) výrazů uvedených níže. Nezapomeňte, že každou
    konstantu $k$ lze vyjádřit jako $kx^0$, kde $x$ jste schopni zaměnit
    za~jakoukoliv jinou proměnnou. Taktéž každou odmocninu jsme schopni zapsat jako
    mocninu.
    \begin{align*}
        &\fracdiff{}{t} \left( \frac38 t^8 \right) &
        &\fracdiff{}{x} \left( \sqrt{x} \right) &
        &\fracdiff{}{z} \left( 2,7z \right) &
        &\fracdiff{}{v} \left( \frac12 mv^2 \right) \\
        &\fracdiff{}{x} \left( \pi \right) &
        &\fracdiff{}{h} \left( 120h \right) &
        &\fracdiff{}{s} \left( 8s^2 \right) &
        &\fracdiff{}{q} \left( 8q \right)
    \end{align*}
\end{exercise}

\subsection{Mocninné funkce geometricky}
\label{subsec:derivace-mocn-fce-geometricky}

Nyní se pokusme řetězové pravidlo zobrazit geometricky. Nejdříve si to ukážeme pro
druhou mocninu.

Nechť máme čtverec se~stranou o~proměnné délce $x$. Obsah tohoto čtverce jsme schopni
spočítat pomocí vzorce. Nechť tedy funkce $S(x) = x^2$ počítá obsah čtverce. Co se
stane, když jisté $x$ zvětšíme o~nepatrnou změnu $\diff{x}$? Poté se nám změní
i~povrch čtverce. Označme změnu plochy čtverce jako $\diff{S}$. Nyní si problém
zobrazíme geometricky. Stranu čtverce jsme zvětšili o~$\diff{x}$ a~nový čtverec
vhodně oddělili na~čtyři části:

\begin{figure}[h!]
    \centering
    \begin{tikzpicture}
    \filldraw[fill=black!10, draw=gray] (0, 0) rectangle (4, -4);
    \draw (2, -2) node[anchor=center] {\huge $S(x)$};
    \filldraw[fill=accentColor!70, draw=gray] (4.1, 0) rectangle (4.5, -4);
    \filldraw[fill=accentColor!70, draw=gray] (0, -4.1) rectangle (4, -4.5);
    \filldraw[fill=accentColor!50!gray, draw=gray] (4.1, -4.1) rectangle (4.5, -4.5);
    \draw [decorate, decoration = {calligraphic brace, amplitude=.3cm}] (4.6, 0) --  (4.6,-4);
    \draw (4.9, -2) node[anchor=west] {$x$};
    \draw [decorate, decoration = {calligraphic brace, amplitude=1mm}] (4.6, -4.1) --  (4.6,-4.5);
    \draw (4.7, -4.3) node[anchor=west] {$\diff{x}$};
    \draw [decorate, decoration = {calligraphic brace, amplitude=.3cm}] (4, -4.6) -- (0, -4.6);
    \draw (2, -4.9) node[anchor=north] {$x$};
    \draw [decorate, decoration = {calligraphic brace, amplitude=1mm}] (4.5, -4.6) -- (4.1, -4.6);
    \draw (4.3, -4.7) node[anchor=north] {$\diff{x}$};
\end{tikzpicture}
    \caption{Geometrická představa čtverce, jehož stranu~$x$ jsme zvětšili o~$\diff{x}$.}
    \label{fig:derivace-rozdel-ctverec}
\end{figure}

Původní čtverec je zobrazen šedě a~nové přírůstky jsou zobrazeny barevně. Takto jsme
schopni schopni vyjádřit změnu obsahu $\diff{S}$ pomocí barevně vyznačených
oblastí v~obrázku~\ref{fig:derivace-rozdel-ctverec} (viz rovnice~\ref{eq:derivace-2mocnina-1}).
Derivaci vyjádříme vydělením~$\diff{x}$ (rovnice~\ref{eq:derivace-2mocnina-2}).
Ale máme menší problém -- přebývá nám tam $\diff{x}$. Protože však $\diff{x}$
chceme co nejblíže k~nule (podle definice derivace pomocí limity), nebude nám vadit,
když $\diff{x}$ vypustíme (rovnice~\ref{eq:derivace-2mocnina-3}) a~funkci~$S(x)$
nahradíme jejím předpisem (rovnice~\ref{eq:derivace-2mocnina-4}):

\begin{subequations}
\begin{align}
    \diff{S} &= 2 \cdot {\color{accentColor} x \, \diff{x}} + {\color{accentColor!50!gray} \diff{x}^2}
    \label{eq:derivace-2mocnina-1} \\
    \fracdiff{S}{x} &= 2x + \diff{x}
    \label{eq:derivace-2mocnina-2} \\
    \fracdiff{S}{x} &= 2x
    \label{eq:derivace-2mocnina-3} \\
    \fracdiff{}{x} \left( x^2 \right) &= 2x
    \label{eq:derivace-2mocnina-4}
\end{align}
\end{subequations}

Jak ověřit pravidlo pro třetí mocninu? Podobně, jen místo čtverce máme krychli. Opět
nám poslouží obrázek krychle, jejíž stranu $x$ jsme zvětšili o~$\diff{x}$ a~vhodně
rozdělili na~osm částí:

\begin{figure}[!ht]
    \centering
    \begin{tikzpicture}[scale=0.5, every node/.style={transform shape}]
    % Původní nezvětšená krychle
    \filldraw[fill=lightgray!90!black] (0, 4) -- (1.5, 5.5) -- (5.5, 5.5) -- (5.5, 1.5) -- (4,0);
    \filldraw[fill=lightgray] (0, 0) rectangle (4, 4);
    \draw (4, 4) -- (5.5, 5.5);

    % Horní oranžový kvádr
    \filldraw[fill=orange!85!black] (0, 6.25) -- (1.5, 7.75) -- (5.5, 7.75) -- (5.5, 7.25) -- (4, 5.75);
    \filldraw[fill=orange] (0, 6.25) rectangle (4, 5.75);
    \draw (4, 6.25) -- (5.5, 7.75);

    % Boční oranžový kvádr
    \filldraw[fill=orange!85!black] (6.25, 0) -- (7.75, 1.5) -- (7.75, 5.5) -- (7.25, 5.5) -- (5.75, 4);
    \filldraw[fill=orange] (6.25, 0) rectangle (5.75, 4);
    \draw (6.25, 4) -- (7.75, 5.5);

    % Přední oranžový kvádr
    \filldraw[fill=orange!85!black] (-1.25, 2.75) -- (-1, 3) -- (3, 3) -- (3, -1) -- (2.75, -1.25);
    \filldraw[fill=orange] (-1.25, 2.75) rectangle (2.75, -1.25);
    \draw (3, 3) -- (2.75, 2.75);

    % Horní boční modrý kvádr
    \filldraw[fill=cyan!50!black] (5.75, 6.25) -- (7.25, 7.75) -- (7.75, 7.75) -- (7.75, 7.25) -- (6.25, 5.75);
    \filldraw[fill=cyan!70!black] (6.25, 6.25) rectangle (5.75, 5.75);
    \draw (6.25, 6.25) -- (7.75, 7.75);

    % Přední boční modrý kvádr
    \filldraw[fill=cyan!50!black] (5, -1.25) -- (5.25, -1) -- (5.25, 3) -- (4.75, 3) -- (4.5, 2.75);
    \filldraw[fill=cyan!70!black] (5, -1.25) rectangle (4.5, 2.75);
    \draw (5.25, 3) -- (5, 2.75);

    % Přední horní modrý kvádr
    \filldraw[fill=cyan!50!black] (-1.25, 5) -- (-1, 5.25) -- (3, 5.25) -- (3, 4.75) -- (2.75, 4.5);
    \filldraw[fill=cyan!70!black] (-1.25, 5) rectangle (2.75, 4.5);
    \draw (3, 5.25) -- (2.75, 5);

    % Malá růžová krychlička
    \filldraw[fill=purple!90!black] (4.5, 5) -- (4.75, 5.25) -- (5.25, 5.25) -- (5.25, 4.75) -- (5, 4.5);
    \filldraw[fill=purple] (4.5, 4.5) rectangle (5, 5);
    \draw (5, 5) -- (5.25, 5.25);

    % Svorky, popisky
    \draw [decorate, decoration = {calligraphic brace, amplitude=1mm}] (2.75, -1.5) -- (-1.25, -1.5);
    \draw (0.75, -2) node[anchor=north] {\huge $x$};

    \draw [decorate, decoration = {calligraphic brace, amplitude=.05cm}] (5, -1.5) -- (4.5, -1.5);
    \draw (4.75, -1.8) node[anchor=north] {\huge $\diff{x}$};
\end{tikzpicture}

    \caption{Krychle, jejíž hranu~$x$ jsme zvětšili o~$\diff{x}$. Změna $\diff{x}$
    je zobrazena velká -- opět kvůli názornosti.}
    \label{fig:derivace-rozdel-krychle}
\end{figure}

Naprosto stejným způsobem jsme schopni vyjádřit změnu objemu krychle pomocí objemu
barevných částí:
\begin{equation*}
    \mathrm{d}V = 3 \cdot {\color{accentColor!50!yellow} x^2 \diff{x}}
        + 3 \cdot {\color{accentColor} x \, \diff{x}^2}
        + {\color{red} \diff{x}^3}
\end{equation*}
