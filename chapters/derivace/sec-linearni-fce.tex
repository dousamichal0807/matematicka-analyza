\section{Diferenciál lineární funkce}
\label{sec:diferencial-linearni-funkce}

Derivace lineární funkce je jednoduchá. Grafem lineární funkce je přímka, u~níž růst
umíme vyjádřit směrnicí, což u~vyjádření pomocí funkčního předpisu odpovídá
koeficientu přítomnému u~lineárního členu.

\begin{lemma}[Derivace lineární funkce]
    \begin{equation*}
        \forall k, q \in \reals: \fracdiff{}{x} \left( kx + q \right) = k
    \end{equation*}
\end{lemma}

Tato definice odpovídá tomu co jsme stanovili
v~oddílu~\ref{subsec:derivace-pomoci-tecen}. Než však budeme pokračovat
se~složitějšími funkcemi, je nutno vypočítat několik příkladů:

\begin{exercise}[Derivace lineárních funkcí]
    Vypočítejte diferenciál následujících lineárních funkcí:
    \begin{align*}
        f(x) &= 2x + 3
        & k(x) &= \frac{3}{2} \left( x + \frac{\pi^2}{7} \right)
        & n(x) &= 5x \sqrt{2} + 3 \sqrt[3]{3} \\
        g(x) &= -4x + \frac{5}{6}
        & \ell(x) &= \frac{5}{7}x + \frac{1}{\pi}
        & p(x) &= k(x) + \ell(x) \\
        h(x) &= -\frac{4}{\pi}x + \frac{\sqrt{2}}{3}
        & m(x) &= \pi x + 2
        & q(x) &= f(x) + g(x) + n(x)
    \end{align*}
\end{exercise}
