\section{Spojitost funkce}
\label{sec:limita-spojitost-fce}

Ještě před začátkem matematické analýzy je nutno vyslovit několik několik lemmat
týkajících se spojitosti funkcí.
Ze~střední školy, možná i~ze~základní, si pamatujete, kdy jste se učili o~intervalech.
Ty nyní budou zásadní -- na~nich postavíme definice a~lemmata o~spojitosti funkcí.
Začneme definicí \emph{okolí} a~\emph{prstencového okolí} bodu a~vyslovíme
\emph{${(\varepsilon, \delta)}$-definici} spojitosti funkce.

\subsection{Okolí bodu}
\label{subsec:limita-spojitost-okoli}

Definice různých druhů okolí bodu nejsou příliš těžké -- tato sekce tedy bude
obsahovat několik definic, ty však budou jednoduché, proto úlohy budou o~něco
pracnější -- bez kalkulačky je spočtete velmi těžko\footnote{Pokud vám připadá, že je
to nemožné, zamyslete se -- matematici před dvěma stoletími neměli kalkulačky --
i~tak dokázali spočítat chytrými technikami například $\pi$ na~mnoho desetinných
míst.}\ldots

\begin{definition}[Okolí bodu]
    $\varepsilon$-okolí bodu~$a$, kde~${a \in \reals}$
    a~${\varepsilon \in \positive{\reals}}$, je značeno~${\ptngb(a; \varepsilon)}$
    a~jedná se o~množinu všech reálných čísel, jejichž vzdálenost od~čísla~$a$ je
    menší než~$\varepsilon$:
    \begin{equation*}
        \ptngb(a; \varepsilon) = (a - \varepsilon;\; a + \varepsilon).
    \end{equation*}
\end{definition}

\begin{exercise}[Okolí bodu]
    Zapište pomocí intervalu a~hraniční body spočítejte s~přesností 5~desetinných
    míst (${e \doteq 2,7182818284\dots}$ je Eulerovo číslo):
    \begin{multicols}{2}
        \begin{enumerate}[label=(\alph*)]
            \item $1,5$-okolí bodu~4
            \item $\pi$-okolí bodu~$e$
            \item $\frac{1}{10}$-okolí bodu~$\sqrt{2}$
            \item $\frac{1}{3\pi}$-okolí bodu~$\frac{20}{\pi}$
            \item ${(2\pi - 4)}$-okolí bodu~${10\pi - 2}$
            \item $0,001$-okolí bodu~$37,157$
        \end{enumerate}
    \end{multicols}
\end{exercise}

\begin{definition}[Levostranné okolí bodu]
    Levostranné $\epsilon$-okolí bodu~$a$kde~${a \in \reals}$
    a~${\varepsilon \in \positive{\reals}}$, je značeno~${\negative\ptngb(a; \epsilon)}$
    a~je jím množina reálních čísel, která jsou menší než~$a$ a~zároveň jsou
    od~bodu~$a$ vzdálena méně než~$\epsilon$:
    \begin{equation*}
        \positive\ptngb(a; \epsilon) = (a - \epsilon;\; a \rangle.
    \end{equation*}
\end{definition}

\begin{exercise}[Levostranné okolí bodu]
    Zapište pomocí intervalu a~hraniční body spočítejte s~přesností 5~desetinných
    míst (${e \doteq 2,7182818284\dots}$ je Eulerovo číslo):
    \begin{enumerate}[label=(\alph*)]
        \item levostranné $0,5$-okolí bodu~$e$
        \item levostranné $\frac{22}{7}$-okolí bodu~$\pi$
        \item levostranné $\frac{1}{10\sqrt{3}}$-okolí bodu~$\sqrt{3}$
        \item levostranné $\frac{1}{5e}$-okolí bodu~$\frac{15}{\pi}$
        \item levostranné $\frac{e}{\pi}$-okolí bodu~$e^\pi$
        \item levostranné $0,002$-okolí bodu~$-1,512$
    \end{enumerate}
\end{exercise}

\begin{definition}[Pravostranné okolí bodu]
    Pravostranné $\epsilon$-okolí bodu~$a$, značené~${\positive\ptngb(a; \epsilon)}$
    je množina
    \begin{equation*}
        \negative\ptngb(a; \epsilon) = \langle a;\; a + \epsilon),
    \end{equation*}
    přičemž musí platit, že ${\epsilon \in \reals^{+}}$ a~${a \in \reals}$.
\end{definition}

To byly definice \emph{okolí bodu}, \emph{pravostranného okolí bodu}
a~\emph{levostranného okolí bodu}.
Rovněž vyslovíme definice i~pro jejich "prstencové varianty".
Prstencové varianty se liší tím, že jejich prvkem není samotný bod, kolem něhož okolí
určujeme, v~našich definicích okolí to byla proměnná~$a$.
Proto lze prstencové okolí napsat jako množinový rozdíl ("neprstencového") okolí bodu
a~jednoprvkové množiny obsahující samotný bod, kolem něhož okolí vzniká.

\begin{definition}[Prstencové okolí]
    Prstencové $\epsilon$-okolí bodu~$a$, které je označováno jako
    ${\ringptngb(a; \epsilon)}$, je množina
    \begin{equation*}
        \ringptngb(a; \epsilon) = \ptngb(a; \epsilon) \smallsetminus \{ a \}
    \end{equation*}
    přičemž musí platit, že ${\epsilon \in \reals^{+}}$ a~${a \in \reals}$.
\end{definition}

\begin{definition}[Levostranné prstencové okolí]
    Levostranné prstencové $\epsilon$-okolí bodu~$a$, které je označováno jako 
    ${\negative\ringptngb(a; \epsilon)}$, je množina
    \begin{equation*}
        \negative\ringptngb(a; \epsilon) = \negative\ptngb(a; \epsilon) \smallsetminus \{ a \}
    \end{equation*}
    přičemž musí platit, že ${\epsilon \in \reals^{+}}$ a~${a \in \reals}$.
\end{definition}

\begin{definition}[Pravostranné prstencové okolí]
    Pravostranné prstencové $\epsilon$-okolí bodu~$a$, které je označováno jako 
    ${\positive\ringptngb(a; \epsilon)}$, je množina
    \begin{equation*}
        \positive\ringptngb(a; \epsilon) = \positive\ptngb(a; \epsilon) \smallsetminus \{ a \}
    \end{equation*}
    přičemž musí platit, že ${\epsilon \in \reals^{+}}$ a~${a \in \reals}$.
\end{definition}

\begin{exercise}[Vztahy mezi okolím a~prstencovým okolím]
    Dokažte, nebo alespoň si uvědomte, proč platí, že:
    \begin{enumerate}[label=(\alph*)]
        \item $\ptngb(a; \epsilon) = \negative\ptngb(a; \epsilon) \cup \positive\ptngb(a; \epsilon)$
        \item $\ringptngb(a; \epsilon) = \negative\ringptngb(a; \epsilon) \cup \positive\ringptngb(a; \epsilon)$
        \item $\positive\ringptngb(a; \epsilon) = (a;\; a + \epsilon)$
        \item $\negative\ringptngb(a; \epsilon) = (a - \epsilon;\; a)$
    \end{enumerate}
\end{exercise}

\subsection{Spojitost funkce v~bodě}
\label{subsec:limita-spojitost-v-bode}

Protože již máme definována okolí a~prstencová okolí bodu, jsme již schopni něco říci
o~spojitosti funkce v~bodě. Začneme definicí, kterou následně rozebereme.

\begin{definition}[Spojitost funkce v~bodě]
    \label{def:funkce-spojitost-v-bode}
    Funkce~$f$ je spojitá v~bodě~$x_0$, pokud pro jakékoliv (sebemenší)
    ${\epsilon \in \positive\reals}$ a~příslušné okolí ${Y = \ptngb(f(x_0); \epsilon)}$
    platí, že pokud~${Y \subset \fndom(f)}$, existuje ${\delta \in \positive\reals}$
    a~k~němu okolí~${X = \ptngb(x_0; \delta)}$ takové, že
    \begin{equation*}
        \forall x \in X : f(x) \in Y
    \end{equation*}
\end{definition}
