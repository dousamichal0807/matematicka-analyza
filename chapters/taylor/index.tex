\begin{chapterintro}{Taylorova řada}{ch:taylor}{red!40!orange!80!gray}{}
\end{chapterintro}

\section{K~čemu se Taylorův rozvoj hodí?}
\label{sec:taylor-uvod}

Všechny funkce nejsou polynomiální -- alespoň ne~na první pohled. \emph{Tayloroův
polynom} resp. \emph{Taylorův rozvoj} je vyjádření jakékoliv funkce pomocí konečného
či nekonečného polynomu. Je-li polynom nekonečný, budeme mu říkat \emph{rozvoj}.
Vyjádření složitých funkcí, jako například exponenciálních či trigonometrických,
pomocí polynomu či rozvoje umožňuje derivovat i~takové funkce.

I~přestože, že před námi bude stát rozvoj, většinou nám vadit nebude. Pomůžeme si
například tím, že rozvoj vyjádříme pomocí operátoru sumace. Většinou v~rozvojích
najdeme určité pravidlo pro~koeficienty jednotlivých členů, tj. dokážeme vyjádřit
koeficient~$c_n$ u~členu~${c_n \cdot x^n}$ pomocí určitého vzorce využívající
proměnnou~$n$. Proto zápis pomocí operátoru sumace nebude příliš těžký.

\section{Trigonometrické funkce}
\label{sec:taylor-trigonometrie}

\begin{equation*}
    \sin(x)
    = \sum_{k=0}^{\infty} \frac{(-1)^k \cdot x^{2k + 1}}{(2k + 1)!}
    = \frac{x^1}{1!} - \frac{x^3}{3!} + \frac{x^5}{5!} - \frac{x^7}{7!} + \cdots
    = x - \frac{x^3}{6} + \frac{x^5}{120} - \frac{x^7}{5040} + \cdots
\end{equation*}

Než však začneme derivovat tento polynom, zkusme uvažovat intuitivně, jak by mohla, alespoň přibližně, derivace vypadat.

V~hodnotě $x = 0$ nám graf $\sin x$ vcelku strmě stoupá, proto bude hodnota derivace funkce pro $x = 0$ jisté kladné číslo. Graf však postupně nestoupá tak strmě, až se stoupání zastaví na~hodnotě $x = \frac12 \pi$. Proto tam derivace funkce bude mít hodnotu~0. Pak graf funkce $\sin(x)$ začne klesat. Nejvíce klesá v~hodnotě $x = \pi$. Zároveň klesá stejně strmě, jako stoupá na~začátku. Proto hodnota derivace funkce pro $x = \pi$ bude mít opačnou hodnotu než hodnota pro $x = 0$. Poté padáme do~záporných hodnot, kde se konečně klesání zastaví v~hodnotě $x = \frac34 \pi$. Tady se nám graf derivace funkce vrací do~hodnoty 0. Abychom ukončili cyklus, pokračujeme, a~graf opět roste. Nakonec se dostáváme na~samotný začátek, kdy graf původní funkce opět roste vcelku strmě, proto zde bude hodnota derviace funkce opět stejná, jako v~bodě $x = 0$.

Napadá vás o~jakou funkci by se mohlo jednat? Pokud ne, tady je řešení:
\begin{equation*}
    \fracdiff{[\sin(x)]}{x} = \cos(x)
\end{equation*}

Zkusme to přes výše zmíněný Taylorův polynom (rozvoj) pro sinus:
\begin{align*}
    \fracdiff{[\sin(x)]}{x}
    &= \fracdiff{}{x} \left(x - \frac{x^3}{6} + \frac{x^5}{120} - \frac{x^7}{5040} + \cdots \right) = \\
    &= \fracdiff{}{x} (x)
    + \fracdiff{}{x} \left(-\frac{x^3}{6}\right)
    + \fracdiff{}{x} \left(\frac{x^5}{120}\right)
    + \fracdiff{}{x} \left(-\frac{x^7}{5070}\right)
    + \cdots \\
    &= 1 - \frac{3x^2}{6} + \frac{5x^4}{120} - \frac{7x^6}{5070} + \cdots \\
    &= 1 - \frac{x^2}{2} + \frac{x^4}{24} - \frac{x^6}{720} + \cdots \\
    &= \cos(x)
\end{align*}

Někdo může namítat, že jsme si $\cos(x)$ vymysleli, a~že~první čtyři členy Taylorova
rozvoje pro $\cos(x)$ jsou shodné s~naším výsledkem, je jen náhoda. Proto pro pokročilejší nabízím verzi s~využitím sumace:
\begin{align*}
    \fracdiff{[\sin(x)]}{x}
    &= \fracdiff{}{x} \left( \sum_{k=0}^{\infty} \frac{(-1)^k \cdot x^{2k + 1}}{(2k + 1)!} \right)
    = \sum_{k=0}^{\infty} \fracdiff{}{x} \left( \frac{(-1)^k \cdot x^{2k + 1}}{(2k + 1)!} \right) \\
    &= \sum_{k=0}^{\infty} \frac{(-1)^k \cdot (2k+1) \cdot  x^{2k}}{(2k+1)!}
    = \sum_{k=0}^{\infty} \frac{(-1)^k \cdot  x^{2k}}{(2k)!}
    = \cos(x)
\end{align*}

\section{Exponenciální funkce}

Jistě všichni slyšeli o~Eulerovu číslu. Pokusme se zderivovat funkci $\exp(x)$, častěji známou jako $e^x$. Opět použijeme Taylorův rozvoj:
\begin{equation*}
    \exp(x) = \sum_{k=0}^{\infty} \frac{x^k}{k!}
    = \frac{x^0}{0!} + \frac{x^1}{1!} + \frac{x^2}{2!} + \frac{x^3}{3!} + \frac{x^4}{4!} + \cdots
    = 1 + x + \frac{x^2}{2} + \frac{x^3}{6} + \frac{x^4}{24} + \cdots
\end{equation*}
