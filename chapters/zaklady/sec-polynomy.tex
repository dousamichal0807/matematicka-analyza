\clearpage

\section{Polynomy}
\label{sec:zaklady-polynomy}

\begin{definition}[Definice polynomu]
    \label{def:polynom}
    Polynom (mnohočlen) je výraz, který lze zapsat jako:
    \begin{equation}
        P(x) = \sum_{k = 0}^{n} c_k x^k = c_0 x^0 + c_1 x^1 + c_2 x^2 +
        \cdots + c_{n-1} x^{n-1} + c_n x^n.
        \label{eq:polynom-obecne}
    \end{equation}
    Čísla ${c_0}$~až~${c_n}$ jsou tzv. \emph{koeficienty polynomu}.
\end{definition}

\subsection{Stupeň polynomu}
\label{subsec:zaklady-polynom-stupen}

\begin{definition}[Stupeň polynomu]
    \label{def:polynom-stupen}
    Nechť existuje polynom~$P$ zapsaný ve~tvaru určeném v~definici~\ref{def:polynom}.
    Poté stupeň polynomu~$P$, který budeme označovat jako ${\deg P}$, bude číslo~$n$
    přítomné v~zápisu polynomu, pokud~${c_n \neq 0}$. Pokud všechny koeficienty
    polynomu~$P$ jsou nulové, stupeň polynomu je roven~$-1$.\footnote{Někdy se v~tomto případě
    určuje, že ${\deg P = -\infty}$.}
\end{definition}

\begin{lemma}
    Pro každý polynom je definován stupeň polynomu.

    \begin{proof}
        Nechť polynom~$P$ je vyjádřen pomocí sumy v~definici~\ref{def:polynom}.
        Jsou-li všechny koeficienty $c_0$~až~$c_n$ rovny nule, podle definice platí,
        že ${\deg P = -1}$. Je-li ${c_n \neq 0}$, opět podle definice platí, že
        ${\deg P = n}$.

        Existuje-li alespoň jeden nenulový koeficient, ale není to koeficient~$c_n$,
        určíme~$m \in \naturals_0$ tak, aby ${c_m \neq 0}$ a~$m$ bylo nejvyšší
        možné s~takovou vlastností. Poté jsme schopni polynom~$P$ přepsat pomocí
        sumace, kde $n$ nahradíme za~$m$. Protože rovnost je zachována
        a~${c_m \neq 0}$, musí platit, že ${m = \deg P}$.
    \end{proof}
\end{lemma}