\begin{chapterintro}{Řešení úloh}{ap:reseni}{gray}{}
    V~tété příloze najdete řešení všech úloh v~této publikaci. Než však řešení úlohy,
    kterou právě řešíte, najdete, zkuste se ještě zamyslet, jestli jste na~něco
    nezapomněli\ldots
\end{chapterintro}

\paragraph{Úloha 3.1.1}
Tečna musí být rovnoběžná s~osou~$y$, abychom nehohli najít její směrnici. Mezi
takové funkce patří např.\ všechny odmocniny s~lichým stupněm počínaje od~tří (tj.
$\sqrt[2n+1]{x}$, kde ${n \in \naturals}$). Grafy lichých odmocnin mají v~bodě
${[0; 0]}$ tečnu rovnoběžnou s~osou~$y$.

\paragraph{Úloha 3.1.2.}
Jedná se například o~některé mocninné funkce, tím jsou na~mysli funkce, které lze
zapsat ve~tvaru ${f(x) = k (x-r)^n + s}$, kde ${k \in \reals \smallsetminus \{0\}}$,
${r \in \reals}$, ${s \in \reals}$ a~${n \in \naturals \smallsetminus \{1\}}$.
Ty mají v~bodě ${[r; s]}$ tečnu, jejíž směrnice je nulová, a~proto i~derivace v~hodnotě
${x = r}$ bude nulová.
Ať zvolíme sebemenší ${\epsilon \in \positive\reals}$, nebudeme moci prohlásit, že
funkce je v~okolí ${\ptngb(r; \epsilon)}$ konstantní (samozřejmě jen tehdy, pokud
vůbec ${\ptngb(r; \epsilon) \subset \fndom(f)}$).

\paragraph{Úloha 3.1.3.}
Derivací funkce~$s(t)$ bude funkce~$v(t)$, což bude okamžitá rychlost v~daném
čase~$t$. Stačí postupovat stejně, jako u~případu okamžité rychlosti a~zrychlení
v~oddílu~\ref{subsec:derivace-jako-podil}.

\paragraph{Úloha 3.1.4.}
