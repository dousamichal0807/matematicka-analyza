\section{Parciální diferenciál}

Problém nastává, máme-li výraz, který obsahuje více proměnných (resp.\ funkci, která má více argumentů). Derivovat můžeme pouze v~závislosti na~jedné proměnné (resp.\ argumentu funkce) a~musíme učinit pro~konkrétní derivaci předpoklad, že zbylé proměnné (resp.\ argumenty funkce) jsou konstantní. Tehdy mluvíme o~tzv. \emph{parciálním diferenciálu}. Navíc v~takovém případě nahrazujeme písmeno~$\text{d}$ znakem~$\partial$.

Pro názornost uvažujme obecnou funkci $f(x, y)$ -- ta má dva argumenty. Poté jsme schopni určit parciální diferenciál zvlášť pro argument $x$ a~zvlášť pro argument $y$:

\begin{align*}
    \frac{\partial f}{\partial x} &= \ldots &
    \frac{\partial f}{\partial y} &= \ldots
\end{align*}

Někdo by mohl namítnout, že existuje implicitní diferenciace. Prozatím implicitní diferenciaci vynechme.