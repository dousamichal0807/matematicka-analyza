%\vfill{}
\pagebreak
\section{Co je vlastně derivace?}

\textit{Derivace}, někdy též \textit{diferenciál}, vyjadřuje, v~jaké míře jistá funkce roste či klesá. Vstupem této operace musí tedy být spojitá funkce. Derivace takové funkce nám jistým způsobem říká, \emph{jakou mírou, jak moc} vstupní funkce roste, či klesá. Nicméně je nutno si říct, jakým způsobem derivace takové věci sděluje.

V~případě, že vstupní funkce $f(x)$ roste v~určitých hodnotách $x$, poté výstupní funkce, označme ji $f'(x)$, která \textit{je derivací} funkce $f(x)$, má pro taková $x$ vždy kladnou hodnotu. Čím více funkce $f(x)$ pro dané $x$ roste, tím větší je hodnota výstupní funkce $f'(x)$ pro stejné $x$. Jako příklad si vezměme stoupající lineární funkci. Vypočteme-li derivaci, výstupem bude funkce, která bude mít kladné hodnoty na~celém svém definičním oboru. Zároveň si povšimněme, že lineární funkce stoupá či klesá neustále stejným tempem, narozdíl od jiných funkcí, např. exponenciálních (stále více roste) či logaritmických (tempo růstu se postupem snižuje). Proto zárověň výstupní funkce bude konstantní.

Opačně se děje pokud vstupní funkce $f(x)$ v~určitých hodnotách $x$ klesá. Tehdy na~výstupní funkci $f'(x)$ je pro taková $x$ hodnota záporná. Opět, čím strměji funkce $f(x)$ klesá, tím zápornější hodnotu bude mít funkce $f'(x)$. Jako příklad nám poslouží opět lineární funkce, samozřejmě klesající. a~jak již asi tušíte, po derivaci nám z ní zbyde záporná konstantní funkce.

Zároveň další důležitou informací je, že v~hodnotách $x$, kde je graf nespojitý, či je zalomen (grafem je lomená čára) není výstupní funkce $f'(x)$ pro taková $x$ definována. Vztah mezi funkcemi $f(x)$ a~$f'(x)$ se zapisuje takto:

\begin{equation*}
    \frac{\text{d}f}{\text{d}x} = f'(x)
\end{equation*}

Nebojte, tento velmi zvláštní zápis si vysvětlíme.