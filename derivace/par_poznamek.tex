\vfill{}
\pagebreak
\section{Pár poznámek}

\paragraph{Zápis diferenciálu složitých výrazů} Prozatím jsme se omezovali jen na~funkce. Můžeme též derivovat (diferenciovat) i~určitý výraz. V~případě, že diferenciovaný (derivovaný) výraz je složitější než jen funkce, můžeme takovou derivaci zapsat několika způsoby. U~všech uvedených způsobů je zvolen stejný výraz k derivaci, aby rozdíl mezi zápisy byly zřejmé (nebojte se -- zatím nebylo řečeno, jak k~takovému výsledku dospět):

\begin{align*}
    \frac{\text{d}}{\text{d}x} \left( \frac12 x - 3 \right) &= \frac12 &
    \frac{\text{d} \left( \frac12 x - 3 \right)}{\text{d}x} &= \frac12 &
    \left( \frac12 x - 3 \right)' &= \frac12
\end{align*}

Všechny tři formy zápisu vyjadřují, že diferenciál výrazu $\frac12 x - 3$ v~závislosti na~$x$ je konstanta rovna $\frac12$. Nejčastěji se využívá možnost vlevo -- umožňuje přehlednější zápis i mnohem složitějších výrazů. Možnost vpravo se nevyužívá, je-li ve vzorci více proměnných (resp.\ argumentů funkce), na~jejichž závislosti jsme schopni derivovat.
