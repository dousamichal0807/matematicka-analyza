\subsection*{Derivace}

\subsubsection*{Neformální definice skrz příklad}
\begin{enumerate}
    \item $\frac{\text{d}s}{\text{d}t} = v(t)$, kde $v(t)$ označuje rychlost. Porovnejte přírůstek ujeté vzdálenosti k~rychlosti podobně jako změnu rychlosti k~zrychlení.
    \item Po nakreslení grafu hybnosti $p(v) = mv$ a~kinetické energie $E_k(v) = \frac12 mv^2$ by mělo být zřejmé, že hybnost je diferenciálem kinetické energie závislé na rychlosti.
\end{enumerate}

\subsubsection*{Derivace mocninných funkcí}
\begin{enumerate}
    \item \begin{itemize}
        \item $\displaystyle \frac{\text{d}}{\text{d}t} \left( \frac38 t^8 \right) = \frac38 \cdot 8t^7 = 3t^7$
        \item $\displaystyle \frac{\text{d}}{\text{d}x} \left( \sqrt{x} \right) = \frac{\text{d}}{\text{d}x} \left( x^{\frac12} \right) = \frac12 x^{\frac12 - 1} = \frac12 x^{-\frac12} = \frac{1}{2 \sqrt{x}}$
        \item $\displaystyle \frac{\text{d}}{\text{d}z} \left( 2,7z \right) = 2,7 \cdot 1 \cdot z^0 = 2,7$
        \item $\displaystyle \frac{\text{d}}{\text{d}v} \left( \frac12 mv^2 \right) = \frac12 m \cdot 2 v^1 = mv$
        \item $\displaystyle \frac{\text{d}}{\text{d}x} \left( \pi \right) = \frac{\text{d}}{\text{d}x} \left( \pi x^0 \right) = \pi \cdot 0x^{-1} = 0$
        \item $\displaystyle \frac{\text{d}}{\text{d}h} \left( 120h \right) = 120 \cdot 1 \cdot h^0 = 120$
        \item $\displaystyle \frac{\text{d}}{\text{d}s} \left( 8s^2 \right) = 8 \cdot 2s^1 = 16s$
        \item $\displaystyle \frac{\text{d}}{\text{d}q} \left( 8q \right) = 8 \cdot 1 \cdot q^0 = 8$
    \end{itemize}
    \item Z~lineárních členů se stala konstanta. Protože míra růstu lineární funkce je po~celou dobu konstantní, derivací musí být konstantní výraz (funkce).
\end{enumerate}